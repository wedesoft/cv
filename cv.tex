%%%%%%%%%%%%%%%%%%%%%%%%%%%%%%%%%%%%%%%%%
% Friggeri Resume/CV
% XeLaTeX Template
% Version 1.0 (5/5/13)
%
% This template has been downloaded from:
% http://www.LaTeXTemplates.com
%
% Original author:
% Adrien Friggeri (adrien@friggeri.net)
% https://github.com/afriggeri/CV
%
% License:
% CC BY-NC-SA 3.0 (http://creativecommons.org/licenses/by-nc-sa/3.0/)
%
% Important notes:
% This template needs to be compiled with XeLaTeX and the bibliography, if used,
% needs to be compiled with biber rather than bibtex.
%
%%%%%%%%%%%%%%%%%%%%%%%%%%%%%%%%%%%%%%%%%

\documentclass[]{friggeri-cv} % Add 'print' as an option into the square bracket to remove colors from this template for printing

\usepackage{thumbpdf}
\usepackage{subfigure}

\addbibresource{bibliography.bib} % Specify the bibliography file to include publications

\renewcommand{\theauthor}{Jan Wedekind}
\newcommand{\thesignature}{\hphantom{\theauthor}}
\newcommand{\thetitle}{Curriculum Vitae}
\newcommand{\thephone}{+44 (793) 133 5290}
\newcommand{\theemail}{\href{mailto:jan@wedesoft.de}{jan@wedesoft.de}}
\newcommand{\thesite}{\href{http://wedesoft.de/}{http://wedesoft.de/}}
\newcommand{\thedate}{04.06.2014}

\newcommand{\renovo}[1]{{\footnotesize Renovo: ``#1''}}

\newcommand{\nref}[2]{#2}

\hypersetup{
   pdftitle      = {\thetitle},
   %pdfsubject    = {Job application as a software engineer at Google},
   pdfsubject    = {Job application as a research engineer in computer vision},
   pdfauthor     = {\theauthor},
   pdfkeywords   = {curriculum vitae, software developer},
   pdfcreator    = {XeTeX, friggeri-cv template},
   pdfproducer   = {pdfTeX-1.40.13},
   bookmarksopen = false,
   colorlinks = false
}


% # CV
% PRESENTATION
% ACHIEVEMENTS
% WORK EXPERIENCE
% RELEVANCE TO THE ROLE
% EDUCATION
% PERSONALITY & SKILLS (TRANSFERABLE) (pashion, hobbies)

% Analytical skills
% Commercial awareness
% Creativity
% Customer focus
% Influence & communication
% Leadership & team work
% planning & organisation
% self-management
% take control of your own learning


% 2-3 pages
% sales pitch <-> what the employer needs
% how you tackle problems, make the most of opportunities
% marketable skills, up-to-date?, know your market
% track record of success
% short-, medium- and long-term goals
% unique selling points
% how are you planning to develop yourself
% easy to read and navigate, have an instanct impact
% clearly shows what you achieved and provide evidence to back it up
% make yout experience relevant to the employer
% evidence of over-achievement

\begin{document}

\header{Jan}{Wedekind}{software developer} % Your name and current job title/field

%----------------------------------------------------------------------------------------
%	SIDEBAR SECTION
%----------------------------------------------------------------------------------------

\begin{aside} % In the aside, each new line forces a line break
\section{contact}
7 William Road
Sutton, SM1 4QT
United Kingdom
~
\thephone
~
\theemail
\href{http://twitter.com/wedesoft}{@wedesoft}
\thesite
~
\section{languages}
German mother tongue
English fluent
\section{programming}
% {\color{red} $\varheartsuit$}
Ruby,
C/C++, GNU Guile,
Java, Clojure,
Python, x86-64 code
\end{aside}

%----------------------------------------------------------------------------------------
%	WORK EXPERIENCE SECTION
%----------------------------------------------------------------------------------------

\section{personal profile}
Forward thinking developer who enjoys problem-solving and working in interdisciplinary teams.
Computer science PhD with specialisation in computer vision and higher-level programming languages.
Main developer of HornetsEye which is the premier computer vision toolkit for the Ruby programming language.
Excellent mathematics and linear algebra skills.
Practical experience with 3D camera calibration and augmented reality.

%\renovo{add personal profile (special skills, relevant to the job).\\
%e.g. experienced/professional software engineer.\\
%Make more readable for HR.}

% 10-15 seconds to get attention

\section{experience}

\begin{entrylist}
%------------------------------------------------
\entry
{2012--2014}
{Digital Science (Macmillan)}
{London, United Kingdom}
{\emph{Software Developer}\\
Commercial software development in small agile teams:% \renovo{more detail}
\begin{itemize}
\item Successfully implemented ``Projects'' GUI using Qt4 and Ruby
\item Introduced automated build process for desktop software
\item Implemented checks against geonames.org data using Clojure and SQLite
\item Held session about fundamentals of computing at company retreat
\end{itemize}}
%------------------------------------------------
\entry
{2004--2009}
{Sheffield Hallam University}
{Sheffield, United Kingdom}
{\emph{part-time Research Assistant} \\
Software development in large interdisciplinary research groups:
\begin{itemize}
\item EPSRC Nanorobotics grant
\begin{itemize}
\item Implemented recognition and tracking algorithms
\item Developed telemanipulation user interface for nano-indenter
\end{itemize}
\item European MiCRoN project
\begin{itemize}
\item Implemented 4-DoF object recognition for microscopy images
\item Developed system architecture for real-time vision software
\end{itemize}
\end{itemize}}
%------------------------------------------------
\entry
{2002-2004}
{Carl Zeiss SMT AG}
{Oberkochen, Germany}
{\emph{Software Developer} \\
2D signal processing of interferometric images, wavelets, and computer generated hologram plots.}
%------------------------------------------------
\end{entrylist}

%----------------------------------------------------------------------------------------
%	EDUCATION SECTION
%----------------------------------------------------------------------------------------

\section{education}

\begin{entrylist}
%------------------------------------------------
\entry
{2005--2012}
{PhD {\normalfont (part-time) in Computer Vision}}
{Sheffield Hallam University}
{\emph{Efficient Implementations of Machine Vision Algorithms using a Dynamically Typed Programming Language}\\
This thesis is about rapid-prototyping of real-time machine vision algorithms using the Ruby programming language.}
%------------------------------------------------
\entry
{1996--2002}
{MSc {\normalfont in computer science}}
{University Fridericiana Karlsruhe}
{Specialisation in compilers, robotics, and measurement engineering}
%------------------------------------------------
\end{entrylist}

%----------------------------------------------------------------------------------------
%	SKILLS SECTION
%----------------------------------------------------------------------------------------

\section{skills}

% TODO: computer science skills
\begin{description}
  \item[computer vision] \href{http://www.wedesoft.de/hornetseye-api/}{HornetsEye} Ruby library\footnote{mature software, main developer}, \href{http://sourceforge.net/projects/mimas}{Mimas} C++ library\footnote{mature software, core contributor}, \nref{http://opencv.org/}{OpenCV}, \href{https://github.com/wedesoft/aiscm}{AIscm} Guile library\footnote{alpha status, main developer}
  \item[image files] \nref{http://www.imagemagick.org/}{ImageMagick}, \nref{http://www.openexr.com/}{OpenEXR}
  \item[audio/video files] \nref{http://www.ffmpeg.org/}{FFmpeg}
  \item[sensors/actors] \nref{http://linuxtv.org/downloads/v4l-dvb-apis/}{V4L2}, \nref{http://damien.douxchamps.net/ieee1394/libdc1394/}{libdc1394}, \nref{https://github.com/OpenKinect/libfreenect}{libfreenect}, \nref{http://www.alsa-project.org/}{ALSA}, \nref{https://github.com/wedesoft/cwiid}{WiiMote}, \nref{https://github.com/wedesoft/robobuilder}{Robobuilder}
  \item[image processing] element-wise operations, histograms, warps, Otsu thresholding, correlation, connected components, edge detection, corner detection, feature descriptors, Hough transform, depth from focus, Fourier transform
  \item[object recognition] normalised cross-correlation, PCA (Camspace algorithm), geometric hashing, RANSAC, planar marker recognition, camera calibration (Zhengyou Zhang)
  \item[object tracking] Camshift, Bounded Hough transform, Lucas-Kanade tracker, SVD matching
  \item[operating systems] \nref{http://www.debian.org/}{Debian} (LPI 101 certification), \nref{http://www.microsoft.com/windows/}{Microsoft Windows} (\nref{http://mingw.org/}{MinGW}), \nref{http://www.apple.com/}{Mac OS X} 
  \item[programming languages] \nref{http://www.gnu.org/software/guile/}{Guile}, \nref{http://www.ruby-lang.org/}{Ruby}, \nref{http://www.python.org/}{Python}, \nref{http://gcc.gnu.org/}{C/C++}, \nref{http://www.intel.com/content/www/us/en/processors/architectures-software-developer-manuals.html}{x86-64 machine code}
  \item[graphical user interfaces] \nref{http://qt-project.org/}{Qt4}, \nref{http://www.opengl.org/}{OpenGL}, \nref{https://en.wikipedia.org/wiki/X\_video\_extension}{X video extension}
  \item[databases] \nref{http://www.mysql.com/}{MySQL}, \nref{http://sqlite.org/}{SQLite}
  \item[build tools] \nref{http://rake.rubyforge.org/}{rake}, \nref{http://www.gnu.org/software/autoconf/}{autoconf}, \nref{http://www.gnu.org/software/automake/}{automake}, \nref{http://www.gnu.org/software/make/}{make}
  \item[software packaging] \nref{http://nsis.sourceforge.net/}{NSIS}, \nref{http://rubygems.org/}{RubyGems}
  \item[version control systems] \nref{http://git-scm.com/}{Git}
  \item[documentation systems] \nref{http://yardoc.org/}{Yardoc}, \nref{http://www.stack.nl/~dimitri/doxygen/}{Doxygen}, \nref{http://naturaldocs.org/}{NaturalDocs}
  \item[testing] \nref{http://rspec.info/}{RSpec}, \nref{http://cukes.info/}{Cucumber}, \nref{http://www.gnu.org/software/automake/manual/html\_node/Introduction-to-TAP.html}{Automake TAP}, \nref{http://jenkins-ci.org/}{Jenkins}, \nref{https://travis-ci.org/}{Travis}
  \item[publishing] \nref{http://www.latex-project.org/}{\LaTeX}, \nref{http://texpower.sourceforge.net/}{TexPower}, \nref{http://www.gnuplot.info/}{GNUPlot}, \nref{http://inkscape.org/}{InkScape}, \nref{http://povray.org/}{PovRay}, \nref{http://www.xfig.org/}{XFig}, \nref{http://www.w3.org/}{XHTML/CSS}
\end{description}

%----------------------------------------------------------------------------------------
%	INTERESTS SECTION
%----------------------------------------------------------------------------------------

\section{interests}

\textbf{professional:} computer vision, functional programming, JIT compilers\\
\textbf{personal:} cooking, reading, fitness, photography % TODO: group activity

% Interests: Computer vision, Scheme (Guile), Clojure, fundamentals of new computing
%----------------------------------------------------------------------------------------
%	PUBLICATIONS SECTION
%----------------------------------------------------------------------------------------

\section{publications}

% \renovo{depending on job send publication list separately}
\printbibsection{article}{articles in peer-reviewed journal}

\printbibsection{inproceedings}{international peer-reviewed conferences/proceedings}

\printbibsection{report}{technical reports}

\printbibsection{misc}{talks/demonstrations}

%----------------------------------------------------------------------------------------

\section{demonstration videos}
\begin{figure}[htbp]
  \begin{center}
    % TODO: fix links for xelatex
    \subfigure[Barcode Scanner]{\label{fig:barcode}\href{https://www.youtube.com/watch?v=Sv31MUMM_EA}{\resizebox{.31\textwidth}{!}{\includegraphics{barcode}}}}\hspace{1em}
    \subfigure[Camshift]{\label{fig:camshift}\href{https://www.youtube.com/watch?v=LBXgXqtt1F8}{\resizebox{.31\textwidth}{!}{\includegraphics{camshift}}}}\hspace{1em}
    \subfigure[Lucas Kanade tracker]{\label{fig:kanade}\href{https://www.youtube.com/watch?v=8VbylCRn3iI}{\resizebox{.31\textwidth}{!}{\includegraphics{kanade}}}}\hspace{1em}
    \subfigure[camera calibration]{\label{fig:calibration}\href{https://www.youtube.com/watch?v=rJVEvBDyVsE}{\resizebox{.31\textwidth}{!}{\includegraphics{calibration}}}}\hspace{1em}
    \subfigure[planar marker]{\label{fig:augmented1}\href{https://www.youtube.com/watch?v=PNTTNAsvH30}{\resizebox{.31\textwidth}{!}{\includegraphics{augmented1}}}}\hspace{1em}
    \subfigure[Augmented Reality]{\label{fig:augmented2}\href{https://www.youtube.com/watch?v=qpFGqTc8O44}{\resizebox{.31\textwidth}{!}{\includegraphics{augmented2}}}}\hspace{1em}
    \subfigure[PCA recognition]{\label{fig:pca}\href{https://www.youtube.com/watch?v=dbY-y44fa_Q}{\resizebox{.31\textwidth}{!}{\includegraphics{pca}}}}\hspace{1em}
    \subfigure[geometric hashing and bounded hough transform]{\label{fig:penguin}\href{https://www.youtube.com/watch?v=zzms8134CHg}{\resizebox{.31\textwidth}{!}{\includegraphics{penguin}}}}\hspace{1em}
    \subfigure[HornetsEye demo]{\label{fig:demo}\href{https://www.youtube.com/watch?v=wNFr7RNWeCs}{\resizebox{.31\textwidth}{!}{\includegraphics{hornetseyedemo}}}}\hspace{1em}
  \end{center}
  \caption{Video demos (Youtube)\label{fig:examples}}
\end{figure}

\end{document}
